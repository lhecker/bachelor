\chapter{Zusammenfassung und Ausblick}

Die Mobilanwendung Mind-Objects erlaubt es einem Nutzer digitale Linienzeichnungen anzufertigen.
Analoge Freihandzeichnungen --- beispielsweise auf Whiteboards --- können jedoch nocht nicht in die Anwendung importiert werden.
Die vorliegende Arbeit dient nun der Erstellung einer Vorgehensweise, sowie darauf aufbauend einer Programmbibliothek, zur Vektorisierung solcher Freihandzeichnungen, damit diese nahtlos in Mind-Objects weiterverarbeitet werden können.

In \autoref{cha:theory} wurde das Problem zunächst theoretisch betrachtet.
Der daraus resultierende Prozess ist in \autoref{fig:theory_flow} zu sehen.
Basierend auf dem Prinzip der Medial-Achsen Transformation~\cite{Blum:1967:ATF} wird eine Fotografie der Zeichnung zunächst binarisiert und dann in eine Skelettierung und Distanztransformation aufgespalten (siehe \autoref{sec:theory_preprocess}).
Bei der Darstellung können Beide zu einer optisch annähernd identischen Repräsentation der binarisierten Grafik wiederhergestellt werden (siehe \autoref{fig:theory_preprocess_reconstructed}).
Darüber hinaus wurde ein Algorithmus entworfen, welcher die Skelettierung in einzelne, polygonale Linienzüge segmentiert (siehe \autoref{sec:theory_segmentation}).
Ebenjene entsprechen somit bereits der internen Repräsentation von Mind-Objects, welche Linienzüge anhand der Mittellinie (hier: Skelettierung) sowie Linienbreite (hier: Distanztransformation) speichert.
Aufgrund der hohen Anzahl von Knotenpunkten der polygonalen Liniensegmente wurde letztendlich, basierend auf dem Douglas-Peucker Algorithmus~\cite{doi:10.3138/FM57-6770-U75U-7727}, eine Methode zur Reduktion der Anzahl an Knotenpunkten, unter Berücksichtigung der Linienbreite, entworfen (siehe \autoref{sec:theory_postprocess}).

Jeder Arbeitsschritt des entworfenen Prozesses wurde als separate Funktion innerhalb einer portablen Programmbibliothek in \texttt{C++} implementiert, deren Aufbau in \autoref{cha:implementation} erläutert wurde.
Für Sauvola's Binarisierungsalgorithmus sowie für das Zhang-Suen Thinning, welche im Entwurf ausgewählt wurden, lagen bereits existierende Implementationen vor.
Diese trugen jedoch in Tests den größten Teil der benötigten Rechenzeit bei, weshalb sie im Rahmen dieser Arbeit neu implementiert wurden (siehe respektiv \autoref{subsec:implementation_preprocess_binarization_optimization} und \autoref{subsec:implementation_preprocess_skeletonization_optimization}).
Die Performanz der entwickelten Implementation wurde dadurch derart verbessert, dass die entwickelte Lösung auch auf langsameren Mobilgeräten zum Einsatz kommen kann.

Anhand von drei exemplarischen Testbildern wurde die optische Qualität der Vektorisierung in \autoref{cha:results} diskutiert, welche für einen ersten Entwurf zufriedenstellend war.
Linienähnliche Gebilde in der Grafik, welche jedoch keine bewussten Linienzeichnungen sind (beispielsweise Schattenbildungen), werden jedoch ebenso vektorisiert.
Da es generell schwierig ist eine Heuristik zu definieren, welche in der Nachbereitung der Vektorgrafik solcherlei ungewollte Linienzeichnungen entfernten könnte, stellt dies ein deutliches Problem dar.
Denn hierdurch fällt die Gesamtlast der optischen Qualität der Vektorisierung einzig auf den Binarisierungsschritt.
Hierfür wurde Sauvola's Methode ausgewählt, denn als Binarisierungsmethode mit lokaler Schwellenfunktion eignet sie sich für Binarisierung bei ungleichmäßiger Beleuchtung (siehe \autoref{sec:essentials_binarization}).
Dieser Umstand konnte in den ausgewählten Testfällen auch positiv überzeugen.
Jedoch hat dies gleichzeitig den Nachteil, dass nur Linien bis zu einer gewissen finiten Breite korrekt vektorisiert werden, was in einigen Fällen zu offensichtlichen, optischen Fehlern führte.

\section{Ausblick}

In \autoref{cha:theory} wurde sich für die Verwendung von Zhang-Suen Thinning entschieden, da der Aufwand zur Implementierung einer korrekten Medial-Achsen Transformation zu aufwändig wäre.
Die Implementation eines modernen Algorithmus dieser Art --- beispielsweise~\cite{DBLP:journals/cg/MonteroL12} --- wäre deshalb eine bedeutende Verbesserung des Vektorisierungsprozesses.
Dies hätte darüber hinaus den Vorteil einer tatsächlich geometrisch korrekten Skelettierung der Grafik, was auch die optische Qualität des Ergebnisses verbessern würde.

Das größte Problem der Vektorisierung stellte allerdings die Qualität der Binarisierung über Sauvola's Methode dar.
Auch die benötigte Rechenzeit ist hierbei problematisch, da diese exponentiell mit der gewählten Fenstergröße steigt, was die maximale Auflösung der zu vektorisierenden Fotografie stark limitiert.
Die Auswahl eines alternativen, potentiell moderneren Prozesses wäre daher angebracht.
Der Umstand, dass Sauvola's Methode auf Graustufengrafiken operiert, stellt darüber hinaus einen deutlichen Nachteil dar, da hierdurch bedeutende Informationen im erweiterten Farbraum verloren gehen.
Publikationen wie~\cite{DBLP:conf/cvpr/Forssen07}, zur Erkennung annähernd gleichfarbiger Blobs in einer Grafik, zeigen, dass dies für eine spätere Erweiterung jedoch möglich sein sollte.
Für eine zukünftige Erweiterung der vorliegenden Arbeit empfiehlt sich deshalb insbesondere eine Verbesserung des Ansatzes zur Aufbereitung der Grafik.
